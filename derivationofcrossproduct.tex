% --------------------------------------------------------------
% This is all preamble stuff that you don't have to worry about.
% Head down to where it says "Start here"
% --------------------------------------------------------------

\documentclass[12pt]{article}

\usepackage[margin=1in]{geometry}
\usepackage{amsmath,amsthm,amssymb,scrextend,commath}
\usepackage{fancyhdr}
\pagestyle{fancy}


\newcommand{\N}{\mathbb{N}}
\newcommand{\Z}{\mathbb{Z}}
\newcommand{\I}{\mathbb{I}}
\newcommand{\R}{\mathbb{R}}
\newcommand{\Q}{\mathbb{Q}}
\renewcommand{\qed}{\hfill$\blacksquare$}
\let\newproof\proof
\renewenvironment{proof}{\begin{addmargin}[1em]{0em}\begin{newproof}}{\end{newproof}\end{addmargin}\qed}
% \newcommand{\expl}[1]{\text{\hfill[#1]}$}

\newenvironment{theorem}[2][Theorem]{\begin{trivlist}
\item[\hskip \labelsep {\bfseries #1}\hskip \labelsep {\bfseries #2.}]}{\end{trivlist}}
\newenvironment{lemma}[2][Lemma]{\begin{trivlist}
\item[\hskip \labelsep {\bfseries #1}\hskip \labelsep {\bfseries #2.}]}{\end{trivlist}}
\newenvironment{problem}[2][Problem]{\begin{trivlist}
\item[\hskip \labelsep {\bfseries #1}\hskip \labelsep {\bfseries #2.}]}{\end{trivlist}}
\newenvironment{exercise}[2][Exercise]{\begin{trivlist}
\item[\hskip \labelsep {\bfseries #1}\hskip \labelsep {\bfseries #2.}]}{\end{trivlist}}
\newenvironment{reflection}[2][Reflection]{\begin{trivlist}
\item[\hskip \labelsep {\bfseries #1}\hskip \labelsep {\bfseries #2.}]}{\end{trivlist}}
\newenvironment{proposition}[2][Proposition]{\begin{trivlist}
\item[\hskip \labelsep {\bfseries #1}\hskip \labelsep {\bfseries #2.}]}{\end{trivlist}}
\newenvironment{corollary}[2][Corollary]{\begin{trivlist}
\item[\hskip \labelsep {\bfseries #1}\hskip \labelsep {\bfseries #2.}]}{\end{trivlist}}

\begin{document}

% --------------------------------------------------------------
%                         Start here
% --------------------------------------------------------------

\lhead{Derivation Of the Cross Product}
\chead{William Emerison Six}
\rhead{\today}

% \maketitle

\begin{problem}{1} %You can use theorem, proposition, exercise, or reflection here.  Modify x.yz to be whatever number you are proving

  Given vectors $\vec{a}$ and $\vec{b}$ in $\R^3$, find a vector in $\R^3$ that is perpendicular both to $\vec{a}$ and to $\vec{b}$.
\end{problem}



\begin{proof}

\textbf{Rotate $\vec{a}$ and $\vec{b}$ so that $\vec{a}$ is on the x axis: }

  Let $\vec{a} = \begin{bmatrix}
    a_1 \\
    a_2 \\
    a_3 \\
  \end{bmatrix}$ and let $\vec{b} = \begin{bmatrix}
    b_1 \\
    b_2 \\
    b_3 \\
  \end{bmatrix}$.

  let $k = \sqrt{a_1^2 + a_2^2}$.

  Define $\vec{f_1}(\vec{c})$ such that $\vec{f_1}(\vec{a})$ is rotated onto the $x$ axis.  Apply $\vec{f_1}$ to $\vec{b}$.

\begin{flalign}
\vec{f_1}(\vec{c}) & = \begin{bmatrix}
     \frac{k}{\norm{a}} & 0  & \frac{a_3}{\norm{a}} \\
     0 & 1  & 0 \\
     \frac{-a_3}{\norm{a}} & 0 & \frac{k}{\norm{a}}\\
    \end{bmatrix} * \begin{bmatrix}
     \frac{a_1}{k} & \frac{a_2}{k} & 0 \\
     \frac{-a_2}{k} & \frac{a_1}{k} & 0 \\
     0 & 0 & 1 \\
\end{bmatrix} * \vec{c} \\
  & = \begin{bmatrix}
     \frac{a_1}{\norm{a}} & \frac{a_2}{\norm{a}} & \frac{a_3}{\norm{a}} \\
     \frac{-a_2}{k} & \frac{a_1}{k} & 0 \\
     \frac{-a_1a_3}{k\norm{a}} & \frac{-a_2a_3}{k\norm{a}} & \frac{k}{\norm{a}} \\
\end{bmatrix} * \vec{c}
\end{flalign}

\begin{flalign}
\vec{f_1}(\vec{b}) = \begin{bmatrix}
     \frac{a_1b_1}{\norm{a}} + \frac{a_2b_2}{\norm{a}} + \frac{a_3b_3}{\norm{a}} \\
     \frac{-a_2b_1}{k}  + \frac{a_1b_2}{k} \\
     \frac{-a_1a_3b_1}{k\norm{a}} + \frac{-a_2a_3b_2}{k\norm{a}} + \frac{kb_3}{\norm{a}} \\
\end{bmatrix}
\end{flalign}



\textbf{Project $\vec{f_1}(\vec{b})$ onto the yz plane }

Define $\vec{f_2}(\vec{c})$ to project any vector $c$ onto the $y-z$ plane.


\begin{flalign}
\vec{f_2}(\vec{c}) & = \begin{bmatrix}
     0 & 0 & 0 \\
     0 & 1 & 0 \\
     0 & 0 & 1 \\
\end{bmatrix} * \vec{c}
\end{flalign}

\begin{flalign}
( \vec{f_2} \circ \vec{f_1}) (\vec{b}) & = \begin{bmatrix}
     0 \\
     \frac{-a_2b_1}{k}  + \frac{a_1b_2}{k} \\
     \frac{-a_1a_3b_1}{k\norm{a}} + \frac{-a_2a_3b_2}{k\norm{a}} + \frac{kb_3}{\norm{a}} \\
\end{bmatrix}
\end{flalign}



\textbf{Rotate $( \vec{f_2} \circ \vec{f_1}) (\vec{b})$ around the x axis by 90 degrees }

Define $\vec{f_3}(\vec{c})$ rotate any vector $c$ around the $x$ axis.

\begin{flalign}
\vec{f_3}(\vec{c}) & = \begin{bmatrix}
     1 & 0 & 0 \\
     0 & 0 & -1 \\
     0 & 1 & 0 \\
\end{bmatrix} * \vec{c}
\end{flalign}

\begin{flalign}
( \vec{f_3} \circ \vec{f_2} \circ \vec{f_1}) (\vec{b}) & = \begin{bmatrix}
     0 \\
     \frac{a_1a_3b_1}{k\norm{a}} + \frac{a_2a_3b_2}{k\norm{a}} + \frac{-kb_3}{\norm{a}} \\
     \frac{-a_2b_1}{k}  + \frac{a_1b_2}{k} \\
\end{bmatrix}
\end{flalign}



\textbf{Rotate the x axis back to $\vec{a}$ }
\begin{flalign}
( \vec{f_1}^{-1} \circ \vec{f_3} \circ \vec{f_2} \circ \vec{f_1}) (\vec{b}) & = \begin{bmatrix}
  \frac{a_1}{\norm{a}} & \frac{-a_2}{k} & \frac{-a_1a_3}{k\norm{a}}\\
  \frac{a_2}{\norm{a}} & \frac{a_1}{k} & \frac{-a_2a_3}{k\norm{a}}  \\
  \frac{a_3}{\norm{a}} & 0 & \frac{k}{\norm{a}} \\
\end{bmatrix} * \begin{bmatrix}
     0 \\
     \frac{a_1a_3b_1}{k\norm{a}} + \frac{a_2a_3b_2}{k\norm{a}} + \frac{-kb_3}{\norm{a}} \\
     \frac{-a_2b_1}{k}  + \frac{a_1b_2}{k} \\
\end{bmatrix} \\
& = \begin{bmatrix}
  \frac{-a_2}{k} * (\frac{a_1a_3b_1}{k\norm{a}} + \frac{a_2a_3b_2}{k\norm{a}} + \frac{-kb_3}{\norm{a}}) + \frac{-a_1a_3}{k\norm{a}} * (\frac{-a_2b_1}{k}  + \frac{a_1b_2}{k})  \\
  \frac{a_1}{k} * (\frac{a_1a_3b_1}{k\norm{a}} + \frac{a_2a_3b_2}{k\norm{a}} + \frac{-kb_3}{\norm{a}}) + \frac{-a_2a_3}{k\norm{a}} * (\frac{-a_2b_1}{k}  + \frac{a_1b_2}{k})  \\ \\
  \frac{k}{\norm{a}} * (\frac{-a_2b_1}{k}  + \frac{a_1b_2}{k}) \\
\end{bmatrix} \\
& = \begin{bmatrix}
  \frac{-a_1a_2a_3b_1}{k^2\norm{a}} + \frac{-a_2^2a_3b_2}{k^2\norm{a}} + \frac{a_2b_3}{\norm{a}} + \frac{a_1a_2a_3b_1}{k^2\norm{a}}  + \frac{-a_1^2a_3b_2}{k^2\norm{a}}  \\
  \frac{a_1^2a_3b_1}{k^2\norm{a}} + \frac{a_1a_2a_3b_2}{k^2\norm{a}} + \frac{-a_1b_3}{\norm{a}} +  \frac{a_2^2a_3b_1}{k^2\norm{a}}  + \frac{-a_1a_2a_3b_2}{k^2\norm{a}}  \\ \\
  \frac{-a_2b_1}{\norm{a}}  + \frac{a_1b_2}{\norm{a}} \\
\end{bmatrix} \\
& = \begin{bmatrix}
  \frac{-a_2^2a_3b_2}{k^2\norm{a}} + \frac{a_2b_3}{\norm{a}} + \frac{-a_1^2a_3b_2}{k^2\norm{a}}  \\
  \frac{a_1^2a_3b_1}{k^2\norm{a}} +  \frac{-a_1b_3}{\norm{a}} +  \frac{a_2^2a_3b_1}{k^2\norm{a}}   \\ \\
  \frac{-a_2b_1}{\norm{a}}  + \frac{a_1b_2}{\norm{a}} \\
\end{bmatrix} \\
& = \begin{bmatrix}
  \frac{-(a_1^2+a_2^2)a_3b_2}{k^2\norm{a}} + \frac{a_2b_3}{\norm{a}}   \\
  \frac{(a_1^2+a_2^2)a_3b_1}{k^2\norm{a}} +  \frac{-a_1b_3}{\norm{a}}    \\ \\
  \frac{-a_2b_1}{\norm{a}}  + \frac{a_1b_2}{\norm{a}} \\
\end{bmatrix} \\
& = \begin{bmatrix}
  \frac{-k^2a_3b_2}{k^2\norm{a}} + \frac{a_2b_3}{\norm{a}}   \\
  \frac{k^2a_3b_1}{k^2\norm{a}} +  \frac{-a_1b_3}{\norm{a}}    \\ \\
  \frac{-a_2b_1}{\norm{a}}  + \frac{a_1b_2}{\norm{a}} \\
\end{bmatrix} \\
& = \begin{bmatrix}
  \frac{-a_3b_2}{\norm{a}} + \frac{a_2b_3}{\norm{a}}   \\
  \frac{a_3b_1}{\norm{a}} +  \frac{-a_1b_3}{\norm{a}}    \\ \\
  \frac{-a_2b_1}{\norm{a}}  + \frac{a_1b_2}{\norm{a}} \\
\end{bmatrix}  \\
& = \frac{1}{\norm{a}} * \begin{bmatrix}
  a_2b_3 -a_3b_2     \\
  a_3b_1 +  -a_1b_3    \\ \\
  a_1b_2 -a_2b_1 \\
\end{bmatrix}
\end{flalign}


\textbf{Scale ( $\vec{f_1}^{-1} \circ \vec{f_3} \circ \vec{f_2} \circ \vec{f_1}) (\vec{b})$ by $\norm{a}$}

\begin{flalign}
  \norm{a} * ( \vec{f_1}^{-1} \circ \vec{f_3} \circ \vec{f_2} \circ \vec{f_1}) (\vec{b})
  & = \norm{a} *  \frac{1}{\norm{a}} * \begin{bmatrix}
  a_2b_3 -a_3b_2     \\
  a_3b_1 +  -a_1b_3    \\ \\
  a_1b_2 -a_2b_1 \\
  \end{bmatrix} \\
  & = \begin{bmatrix}
  a_2b_3 -a_3b_2     \\
  a_3b_1 +  -a_1b_3    \\ \\
  a_1b_2 -a_2b_1 \\
\end{bmatrix}
\end{flalign}
\end{proof}


\end{document}
